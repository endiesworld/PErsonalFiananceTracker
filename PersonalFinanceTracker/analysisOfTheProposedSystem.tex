\chapter{Analysis of the Proposed System \\
\small{\textit{-- Emmanuel Okoro and Spurthi Setty}}
\index{Analysis of the Proposed System} 
\index{Chapter!Analysis of the Proposed System}
\label{Chapter::Analysis of the Proposed System}}


\section{Summary of Improvements}
The Personal Finance Tracker (PFT) app introduces several improvements over traditional and existing financial tracking methods:

\begin{enumerate}
    \item \textbf{Real-Time Financial Tracking}: The app supports real-time updates through a combination of webhook integrations and periodic pulls, eliminating the need for manual syncing of financial data.

    \item \textbf{Automated Data Collection and Entry}: Users can avoid tedious manual entry of transactions. The app can connect directly to financial institutions and scan grocery receipts to extract itemized data.

    \item \textbf{AI-Driven Recommendations}: By leveraging machine learning, the app can provide personalized financial insights, including spending analysis, budget optimization, and investment performance tracking.

    \item \textbf{Enhanced Security}: End-to-end encryption, secure cloud infrastructure, and adherence to financial data protection regulations ensure user data safety.

    \item \textbf{User-Centric Design}: Features such as customizable dashboards, zero-based budgeting tools, and gamified spending goals enhance user engagement and satisfaction.

    \item \textbf{Comprehensive Financial Planning}: The app integrates multiple budgeting frameworks and offers advanced tools like refund tracking and tax calculation, empowering users to make well-informed financial decisions.
\end{enumerate}

\section{Disadvantages and Limitations}

\begin{enumerate}
    \item \textbf{Dependence on External Services}: The app relies on financial institution APIs and webhook integrations for real-time data. Any outages or API limitations from these services could impact functionality.

    \item \textbf{Initial Setup Complexity}: Users may need to invest time initially to link their accounts, configure budgets, and customize settings.

    \item \textbf{Privacy Concerns}: Despite robust security measures, some users may hesitate to share sensitive financial information with the app.

    \item \textbf{AI Prediction Errors}: AI-driven insights may not always be accurate or suitable for all users, particularly in niche financial situations.

    \item \textbf{Internet Dependency}: Real-time updates and cloud-based features require a stable internet connection, limiting usability in offline scenarios.

    \item \textbf{Learning Curve}: While designed to be user-friendly, users unfamiliar with financial apps might find certain advanced features overwhelming initially.
\end{enumerate}

\section{Alternatives and Trade-Offs Considered}

\begin{enumerate}
    \item \textbf{Manual Data Entry vs. Automated Syncing}:
    \begin{itemize}
        \item \textit{Alternative}: Allow users to enter all financial data manually.
        \item \textit{Trade-Off}: While this provides more control to users, it increases the effort required, reducing engagement and user retention.
    \end{itemize}

    \item \textbf{Periodic Updates vs. Real-Time Updates}:
    \begin{itemize}
        \item \textit{Alternative}: Implement only periodic updates to simplify implementation.
        \item \textit{Trade-Off}: Periodic updates are less resource-intensive but lack the immediacy and convenience of real-time updates.
    \end{itemize}

    \item \textbf{Local Data Storage vs. Cloud-Based Storage}:
    \begin{itemize}
        \item \textit{Alternative}: Store all data locally on user devices to enhance privacy.
        \item \textit{Trade-Off}: While local storage improves privacy, it limits accessibility and could lead to data loss if devices are damaged or lost.
    \end{itemize}

    \item \textbf{AI Features vs. Rule-Based Recommendations}:
    \begin{itemize}
        \item \textit{Alternative}: Use predefined rules for recommendations instead of AI.
        \item \textit{Trade-Off}: Rule-based recommendations are simpler but lack the personalization and adaptability of AI-driven insights.
    \end{itemize}

    \item \textbf{Simplified Features vs. Comprehensive Tools}:
    \begin{itemize}
        \item \textit{Alternative}: Focus only on basic financial tracking and exclude advanced features like tax calculations or investment tracking.
        \item \textit{Trade-Off}: While simpler features reduce development complexity, they also limit the app’s appeal to users with diverse financial needs.
    \end{itemize}
\end{enumerate}





