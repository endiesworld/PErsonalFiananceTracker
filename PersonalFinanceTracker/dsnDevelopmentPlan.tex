\chapter{Development Plan \\
\small{\textit{-- Author Name}}
\index{development plan} 
\index{Chapter!Development Plan}
\label{Chapter::DevelopmentPlan}}

\section{Introduction (required)}

This is a brief description of the project, at most 2 paragraphs.  You should point to key documents, e.g., requirements documents, architecture review documents, state the problem you are trying to solve and state success criteria. Briefly state what is the problem you are trying to solve and perhaps a user story of how it will be used.

\section{Roles and Responsibilities (required)}
These vary for each type of product. For small projects, folks may serve multiple roles.  This is a list of common roles we have used for software development: 
 
\begin{enumerate}
\item Development Lead (only 1 name) 
\item Buildmeister (only 1 name) 
\item Architect (only 1 name) 
\item Developers (multiple) 
\item Test Lead (only 1 name) 
\item Testers (multiple) 
\item Documentation (multiple, usually ALL) 
\item Documentation Editor (only 1 name) 
\item Designer (only 1 name) 
\item User advocate (only 1 name) 
\item Risk Management (only 1 name) 
\item System Administrator (only 1 name) 
\item Modification Request Board (1 leader, multiple representatives) 
\item Requirements Resource (usually 1 name) 
\item Customer Representative (multiple) 
\item Customer responsible for acceptance testing
\end{enumerate}

\section{Method (required)}
These are unique to software development, although there may be some overlap.

\subsection{Software}
\begin{enumerate}
\item Language(s) with version number including the compiler if appropriate 
\item Operating System(s) with release number 
\item Software packages/libraries used with release/version number 
\item Code conventions – this should preferably be a pointer to a document agreed to and followed by everyone 
\end{enumerate}

\subsection{Hardware}
\begin{enumerate}
\item Development Hardware 
\item Test Hardware 
\item Target/Deployment Hardware 
\end{enumerate}

\subsection{Backup plan (individual and project)} 

Risk management.

\subsection{Review Process} 
\begin{enumerate}
\item Will you do architecture, usability, design, security, privacy or code reviews?
\item What approach will you use for the reviews (formal, informal, corporate standard)? 
\item Who is responsible for the reviews and resolving any issues uncovered by the reviews? 
\item Code readings? 
\end{enumerate}

\subsection{Build Plan} 
\begin{enumerate}
\item Revision control system and repository used
\item Continuous integration 
\item Regularity of the builds – daily 
\item Deadlines for the builds – deadline for source updates 
\item Multiplicity of builds 
\item Regression test process – see test plan 
\end{enumerate}

\subsection{Modification Request Process} 
\begin{enumerate}
\item MR tool 
\item Decision process (board – if more than paragraph should point to alternate description) 
\item State whether there will be two process streams one during development and one after development 
\end{enumerate}

\section{Virtual and Real Workspace}
It is great to have a project room, it is also great to use a wiki or some document repository system so long as it is private.  Google docs is not private but may be sufficient for a class or university project.

\section{COMMUNICATION PLAN (required)}

\subsection{Heartbeat Meetings}
This section describes the operation of the “heartbeat” meetings, meetings that take the pulse of the project.  Usually, these meetings are weekly, and I prefer to have them early in the day before folks get into their regular routine, but this is not necessary.  The meeting should include only necessary individuals – no upper-level management or lurkers.  It should have a set agenda, with the last part of the meeting reviewing open issues and risks.  It should be SHORT, thirty minutes or less is ideal.  Notes should be provided after the meeting and issues should be tracked and reviewed each meeting, usually at the end.

\subsection{Status Meetings}
Status meetings have management as their target and should be held less frequently than 
heartbeat meetings, preferably biweekly, monthly or quarterly depending on the duration 
of the project.  It is solely to provide status for the project.  If issues arise, they should be 
addressed at a separate meeting (see next item). These should be short.  This section 
should describe the format and periodicity.

\subsection{Issues Meetings}
If a problem does arise, never surprise your manager.  Schedule a meeting at his or her 
earliest convenience.  This section describes how alerts will arise and the governance of 
when to trigger an alert – usually after a discussion at the heartbeat meeting.

\section{Timeline and Milestones (required)}

This section should be crisp containing 4-10 milestones for the duration of the project, 
each of which would trigger a re-issuing of this document to report on progress.  Each 
milestone should define a 100% complete item, should list the critical participants and 
list begin time and end time.  Each time you re-issue this document you should highlight 
changes with italics or bold with the track change features – colors will not show up on a photocopy.

\section{Testing Policy/Plan (required)}
This section should describe your testing methodology, e.g., you may include your plans for:
\begin{enumerate}
\item	Test driven development
\item	Unit testing
\item	Integration testing 
\item	System testing 
\item	Acceptance testing 
\item	Regression testing
\item	Testing for critical quality attributes
\item	Any frameworks being used for your testing
\end{enumerate}

At least indicate if certain type of testing will be practiced, if so, when to start, and what is the stopping criteria. For example, unit testing is recognized as important, and recommended. But sometimes start-up companies intentionally skip unit testing for speed to market.

\section{Risks (required)}
Ideally it occurs because of an established risk management program.  If that does not exist, do the best you can to enumerate the risks and explain how they will be track, monitored, and mitigated.

\section{Assumptions (required)}
It may be clear to the project insiders what assumptions are being made about staffing, hardware, vacations, rewards, ... but make it clear to everyone else and to the other half of the project that cannot read your thoughts.

\section{DISTRIBUTION LIST}
 
Who receives this document?

\section{IRB Protocol (required)}
Does your project need IRB application. 
Refer to \\
\url{https://sit.instructure.com/courses/67496/pages/irb-information?module_item_id=1960974}


\section{Required Resource and Budget (required)}

Describe what resources, hardware, software, and other resources, you need for your project. Include a tentative budget for your resource. The more accurate the better.

\section{Worry Beads (optional)}  
This section describes the things as manager I am most worried about at the time of latest document issue.  This section is useful because it helps you to focus on the parts most likely to fail.  Sometimes, I segment the worries by time scale: day, week, month, 
quarter ... lifetime. 

\section{Documentation Plan (optional)}  
Many years ago we had much too much documentation, now we have precious little – 
this must change.  Write documentation as if you’ll need to personally support the 
project forever – you just might need to and you’ll be glad you took the time to document the obvious, the not so obvious and the obscure.  As an example, it’s useful to document alternate architectures and designs you did not pursue along with the rationale.  What were the “gotchas” you were trying to avoid?   

\section{Build Plan (optional)} 
When builds and testing become complex, this might be a separate section or point to a 
separate document. 

\section{Others (optional)}
Other aspects that are relevant to your project
Any other aspects that are important to your project but not listed above, you can add them to new sections.
