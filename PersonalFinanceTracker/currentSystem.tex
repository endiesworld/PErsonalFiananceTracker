\chapter{Current System \\
\small{\textit{-- Emmanuel Okoro and Spurthi Setty}}
\index{Current System} 
\index{Chapter!Current System}
\label{Chapter::Current System}}

\section{Current System or Situation}

\subsection{Background, Objectives, and Scope}
Several personal finance applications are currently available, each offering unique approaches to financial management. These include:
\begin{enumerate}
    \item \textbf{YNAB (You Need a Budget):} A zero-based budgeting tool requiring users to allocate all income to specific categories. While it integrates with bank accounts, it still relies heavily on manual data entry and transaction categorization.
    \item \textbf{Goodbudget:} Employs the envelope budgeting method, requiring users to manually assign portions of their income to spending categories. This app does not sync with bank accounts, necessitating manual input of balances and transactions.
    \item \textbf{EveryDollar:} Provides a simpler zero-based budgeting model, offering manual entry in its free version and bank account syncing in its premium version. However, it lacks advanced features like tax calculations and investment tracking.
\end{enumerate}

While these systems have been effective for users committed to manual financial management, they exhibit several shortcomings:
\begin{itemize}
    \item Time-consuming manual data entry.
    \item Limited automation for receipt scanning or financial forecasting.
    \item Lack of integration with tax filing or investment tracking.
\end{itemize}

The Personal Finance Tracker (PFT) seeks to overcome these limitations by providing a more automated, AI-powered solution. Its objective is to deliver a streamlined experience by combining real-time bank syncing, receipt scanning, and advanced financial insights to enhance users’ financial decision-making.

\subsection{Operational Policies and Constraints}
Existing systems have the following operational policies and constraints:
\begin{itemize}
    \item \textbf{Manual Entry Requirements:} Applications like Goodbudget and YNAB require users to manually input transaction details, which may lead to inaccuracies and inconsistencies.
    \item \textbf{Limited Operating Hours:} Systems relying on manual user input are limited by user availability.
    \item \textbf{Hardware Constraints:} Current systems primarily operate on user-provided devices such as smartphones or personal computers without leveraging advanced cloud infrastructure for scalability.
    \item \textbf{Security Concerns:} Systems like YNAB and EveryDollar integrate with financial accounts but may lack robust privacy measures for sensitive data.
    \item \textbf{Cost of Operations:} Premium features in these apps often require subscription fees, limiting accessibility for cost-sensitive users.
\end{itemize}

\subsection{Description of the Current System or Situation}

\textbf{Operational Environment:} Current personal finance systems operate on mobile and web platforms. Users interact with these apps via their devices, with data processing often restricted to local storage or limited cloud functionality.

\textbf{Major Components and Interconnections:}
\begin{itemize}
    \item \textbf{User Interfaces:} Mobile and web apps for data input and reporting.
    \item \textbf{Integration Modules:} Connect with bank accounts for limited transaction syncing (e.g., EveryDollar premium).
    \item \textbf{Storage Solutions:} Local or basic cloud-based databases.
\end{itemize}

\textbf{Capabilities and Features:}
\begin{itemize}
    \item Zero-based and envelope budgeting methods.
    \item Manual transaction categorization and budget adjustments.
    \item Limited reporting tools for tracking expenses and income.
\end{itemize}

\textbf{Performance Characteristics:}
\begin{itemize}
    \item Manual data entry results in slow processing times.
    \item Limited scalability due to reliance on user devices.
    \item Inconsistent data accuracy caused by user errors.
\end{itemize}

\textbf{Operational Risks:}
\begin{itemize}
    \item High dependency on user discipline for consistent data entry.
    \item Risk of data breaches due to insufficient encryption and security protocols.
    \item Potential loss of data during device failures or app unavailability.
\end{itemize}

\textbf{Quality Attributes:}
\begin{itemize}
    \item \textbf{Availability:} Dependent on the user’s device and app stability.
    \item \textbf{Usability:} Interfaces are often simple but require high user engagement.
    \item \textbf{Maintainability:} Updates focus on bug fixes and minor enhancements.
    \item \textbf{Security:} Lacks advanced encryption and may not comply fully with privacy regulations.
\end{itemize}

\textbf{Safety, Security, and Privacy Provisions:}
\begin{itemize}
    \item Basic encryption for user data.
    \item Limited privacy controls, often relying on user discretion.
    \item No significant provisions for operational continuity in emergencies.
\end{itemize}

\subsection{Modes of Operation for the Current System}
\begin{itemize}
    \item \textbf{Manual Mode:} Utilized by apps like Goodbudget and YNAB for manual data entry and management.
    \item \textbf{Sync Mode:} Apps like EveryDollar’s premium version offer limited bank syncing, requiring manual transaction categorization.
    \item \textbf{Degraded Mode:} Limited functionality during connectivity issues, often requiring full user intervention to update records later.
\end{itemize}

\subsection{User Classes and Other Involved Personnel}

\subsubsection{Organizational Structure}
\begin{itemize}
    \item \textbf{End Users:} Individuals managing personal finances via these tools.
    \item \textbf{Support Personnel:} Provide technical support and user assistance for app-related issues.
\end{itemize}

\subsubsection{Profiles of User Classes}
\begin{itemize}
    \item \textbf{Budget-Conscious Users:} Individuals seeking to manage income and expenses manually.
    \item \textbf{Tech-Savvy Users:} Prefer tools with bank syncing and automated features.
    \item \textbf{Occasional Users:} Individuals who use the app sporadically for specific financial planning tasks.
\end{itemize}

\subsubsection{Interactions Among User Classes}
Users primarily interact with the app interfaces, while support personnel address technical queries or resolve issues. Developers interact indirectly through system updates and feature rollouts. Minimal collaboration exists between user classes in current systems.

\subsubsection{Other Involved Personnel}
\begin{itemize}
    \item \textbf{App Developers:} Design and maintain the software.
    \item \textbf{Data Analysts:} Analyze user data for product improvement (if applicable).
    \item \textbf{Policy Makers:} Ensure compliance with financial and privacy regulations.
\end{itemize}

\subsection{Support Environment}
\begin{itemize}
    \item \textbf{Support Agencies:} Often minimal, with user-reported issues addressed via FAQs or limited customer support teams.
    \item \textbf{Maintenance:} Regular updates for bug fixes and minor feature enhancements.
    \item \textbf{Storage and Backup:} Basic cloud storage for user data but often lacking robust backup systems.
    \item \textbf{Repair and Replacement Criteria:} Dependent on the user’s device health; no significant system-level provisions for repair.
\end{itemize}






