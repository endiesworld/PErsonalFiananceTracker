\chapter{Justification for and Nature of Changes \\
\small{\textit{-- Emmanuel Okoro and Spurthi Setty}}
\index{Justification for and Nature of Changes} 
\index{Chapter!Justification for and Nature of Changes}
\label{Chapter::Justification for and Nature of Changes}}

\section{Justification for and Nature of Changes}
\subsection{Justification for Changes}
The current methods of managing finances are inadequate to provide real-time information, automation, or security. There is a need for a more secure, automated and insightful financial management system.

\subsubsection{Summary of New or Modified Aspects}
\begin{itemize}
    \item \textbf{Enhanced Automation}: Existing solutions require extensive manual data entry, leading to inefficiencies and errors. PFT addresses this by automating transaction imports, receipt scanning, and tax calculations.
    \item \textbf{Advanced Insights}: The system provides real-time financial forecasting and actionable insights powered by AI, which are absent in most competitor apps.
    \item \textbf{Integrated Functionalities}: Unlike competitors that focus on singular budgeting approaches, PFT integrates zero-based budgeting, envelope budgeting, and simplified frameworks into a unified platform.
    \item \textbf{Enhanced Security}: The proposed system incorporates two-factor authentication (2FA) and encryption, addressing increasing concerns about financial data privacy and security.
\end{itemize}

\subsubsection{Deficiencies of the Current System}
\begin{itemize}
    \item \textbf{Manual Data Entry}: Current apps like YNAB and Goodbudget rely heavily on manual input, which is time-consuming and prone to human error.
    \item \textbf{Limited Automation and Integration}: Most competitors lack advanced automation, such as real-time receipt scanning or comprehensive tax integration.
    \item \textbf{Narrow Feature Set}: Existing platforms often fail to provide holistic financial management tools, such as combined investment tracking, tax assistance, and customizable cash flow projections.
\end{itemize}

\subsubsection{Justification for a New or Modified System}
\begin{enumerate}
    \item \textbf{New Opportunity}: The rise in demand for automated and intelligent personal finance tools presents a clear opportunity. Developing a new system that leverages advanced AI and automation will fill this market gap and provide users with unprecedented convenience and accuracy.
    \item \textbf{Operational Improvement}: By automating tedious manual processes and offering AI-powered insights, PFT significantly improves user efficiency and decision-making. This enhancement reduces the lifecycle cost of financial management for end-users and increases engagement.
    \item \textbf{Necessity of New Functional Capabilities}: The inclusion of tax filing assistance and detailed cash flow forecasting introduces essential functionalities that empower users to achieve their financial goals effectively. These capabilities align with modern user expectations and differentiate PFT from its competitors.
\end{enumerate}


% \section{Description of Desired Changes}
% \begin{itemize}
%     \item \textbf{Automation of financial tracking:} Integration with banks to automate the import of financial transactions.
%     \item \textbf{Real-time recommendations:} Provide users with actionable insights to help meet financial goals.
%     \item \textbf{Security enhancements:} Implement two-factor authentication (2FA) and data encryption.
% \end{itemize}

\section{Description of Desired Changes}
\subsubsection{Capability Changes}
\begin{itemize}
    \item \textbf{Automated Financial Tracking}: Integration with banks to automate the import of financial transactions in real time.
    \item \textbf{Tax Assistance Features}: Inclusion of tools for calculating and filing tax returns, currently absent in competitor systems.
    \item \textbf{Investment Tracking}: New functionality for tracking and forecasting investment performance.
    \item \textbf{Real-time recommendations:} Provide users with actionable insights to help achieve financial goals.
    \item \textbf{Receipt Scanning:} Use AI to extract data from receipts for automatic entry.
    \item \textbf{Security enhancements:} Implement two-factor authentication (2FA) and data encryption.
\end{itemize}


\subsubsection{Interface Changes}
\begin{itemize}
    \item Improved user interface with seamless navigation for both mobile and web platforms.
    \item New interfaces for AI-driven features, such as receipt scanning and tax preparation.
    \item Enhanced integration capabilities with external APIs for banking and investment platforms.
\end{itemize}

\section{Priorities Among Changes}
\begin{itemize}
    \item \textbf{Essential:} Automated Financial Tracking, security features (encryption, 2FA), Real-Time recommendations
    \item \textbf{Desirable:} Tax Assistance Features, User-friendly interfaces, Receipt Scanning, New Interfaces for AI-Driven features
    \item \textbf{Optional:} Investment tracking, enhanced API integration
\end{itemize}


\section{Changes considered but not included}
\begin{itemize}
    \item Detailed Investment Insights 
    \item Tax Filing support for small businesses 
\end{itemize}


\section{Assumptions and constraints}

*include key constraints from chapter 9* 
