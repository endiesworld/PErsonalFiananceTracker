\chapter{Concepts for the Proposed System \\
\small{\textit{-- Emmanuel Okoro and Spurthi Setty}}
\index{glossary} 
\index{Chapter!Concepts for the Proposed System}
\label{Chapter::Concepts for the Proposed System}}

\section{Concepts for the Proposed System}

\subsection{Background, Objectives, and Scope}
The Personal Finance Tracker (PFT) aims to empower users by providing a comprehensive and automated tool for financial management. It addresses the limitations of manual financial tracking by integrating cutting-edge technologies such as real-time data syncing, AI-powered insights, and robust security protocols.

\textbf{Objectives:}
\begin{itemize}
    \item Automate income and expense tracking.
    \item Offer personalized financial insights and recommendations.
    \item Ensure high levels of data security and privacy compliance.
    \item Provide tools to streamline budgeting, savings, and investment planning.
\end{itemize}

\textbf{Scope:}
The PFT will cater to a diverse user base, including individuals and small business owners, operating across mobile and web platforms. It integrates with financial institutions, offers receipt scanning capabilities, and delivers insights via AI models to optimize financial decision-making.

\subsection{Operational Policies and Constraints}
\begin{itemize}
    \item The system must comply with data protection regulations such as GDPR and CCPA.
    \item Operations are constrained by budgetary and timeline considerations, requiring modular and iterative development approaches.
    \item The app must be available 24/7 with minimal downtime, supported by cloud-based infrastructure and backup systems.
    \item Limited to integration with financial institutions that provide open banking APIs.
    \item Emergency response plans must be in place to handle unexpected data breaches or service outages.
\end{itemize}

\subsection{Description of the Proposed System}

\textbf{Operational Environment:}
The system operates across Android, iOS, and web platforms. Users access the app via secure connections supported by scalable cloud infrastructure. Backup data centers ensure continuity in case of primary server failures.

\textbf{Major System Components:}
\begin{itemize}
    \item Mobile and Web Interfaces: For user interaction and financial data visualization.
    \item Backend API: Handles transaction data processing, analytics, and communication with external systems.
    \item Database: Cloud-hosted, encrypted storage for user data and transactional records.
    \item Integration Modules: APIs for banking synchronization, receipt scanning, and third-party financial services.
    \item Notification System: Alerts users about spending patterns, overdue payments, and budgeting milestones.
\end{itemize}

\textbf{Key Features:}
\begin{itemize}
    \item Real-time tracking of transactions and account balances.
    \item Budget creation and savings goal tracking.
    \item Receipt scanning with AI-based item recognition.
    \item Automated tax calculation and reporting tools.
    \item Personalized insights and recommendations based on user data.
    \item Data encryption, two-factor authentication (2FA), and privacy controls.
\end{itemize}

\textbf{Performance Characteristics:}
\begin{itemize}
    \item High availability with 99.9\% uptime.
    \item Data processing speed optimized for real-time updates.
    \item Support for thousands of concurrent users with scalable cloud infrastructure.
\end{itemize}

\textbf{Quality Attributes:}
\begin{itemize}
    \item \textbf{Reliability:} Built-in redundancy and disaster recovery systems ensure continuity.
    \item \textbf{Usability:} Intuitive interfaces tailored for users of varying financial literacy levels.
    \item \textbf{Security:} End-to-end encryption, compliance with industry standards, and regular security audits.
    \item \textbf{Interoperability:} Seamless integration with third-party APIs and banking systems.
    \item \textbf{Scalability:} The system is designed to scale with increasing user demand.
\end{itemize}

\subsection{Modes of Operation}
\begin{itemize}
    \item \textbf{Normal Mode:} Users track financial activity, set goals, and receive real-time insights.
    \item \textbf{Degraded Mode:} The system allows manual data input and access to cached information during connectivity issues.
    \item \textbf{Emergency Mode:} Data backup and recovery mechanisms are activated to restore operations in case of disasters.
    \item \textbf{Maintenance Mode:} Administrators perform updates, patches, and system diagnostics to ensure ongoing reliability.
\end{itemize}

\subsection{User Classes and Other Involved Personnel}

\subsubsection{Organizational Structure}
\begin{itemize}
    \item Users: Individuals and small business owners using the app for personal and professional financial management.
    \item System Administrators: Oversee infrastructure, manage permissions, and ensure system integrity.
    \item Banking Integration Specialists: Maintain and troubleshoot connections with external financial APIs.
    \item Security Specialists: Monitor and ensure the app adheres to security best practices.
\end{itemize}

\subsubsection{Profiles of User Classes}
\begin{itemize}
    \item \textbf{Individuals:} Manage personal budgets, track expenses, and optimize savings.
    \item \textbf{Small Business Owners:} Use the app for expense categorization, tax preparation, and financial planning.
    \item \textbf{System Operators:} Oversee app health, user reports, and manage system workflows.
\end{itemize}

\subsubsection{Interactions Among User Classes}
Users primarily interact with the app, while administrators manage backend operations and integrations. Collaboration between user support, developers, and administrators ensures continuous improvement. Security teams interact indirectly to maintain compliance and resolve vulnerabilities.

\subsubsection{Other Involved Personnel}
\begin{itemize}
    \item \textbf{Executive Stakeholders:} Oversee system performance, user satisfaction, and business alignment.
    \item \textbf{Policy Makers:} Ensure that the app adheres to financial and data protection regulations.
    \item \textbf{User Clients:} Indirect beneficiaries of the system, such as users’ financial advisors or auditors.
\end{itemize}

\subsection{Support Environment}
\begin{itemize}
    \item \textbf{Support Agencies:} Dedicated teams for technical support and user assistance.
    \item \textbf{Facilities and Equipment:} Cloud-based infrastructure for storage, processing, and scalability.
    \item \textbf{Maintenance:} Regular updates, patches, and periodic performance evaluations to ensure optimal functionality.
    \item \textbf{Backup Systems:} Automated data backup and recovery solutions to safeguard user information.
    \item \textbf{Documentation and Training:} Comprehensive guides and training programs for users, administrators, and support staff.
\end{itemize}

\textbf{Cost of Operations:}
Costs will include cloud infrastructure fees, API subscription fees for financial integrations, AI model training and deployment, and ongoing maintenance by the development team. Scalability measures ensure cost efficiency relative to user growth.

\textbf{Operational Risk Factors:}
\begin{itemize}
    \item Risk of system outages mitigated by high-availability architecture.
    \item Potential data breaches addressed by strong encryption and regular security audits.
    \item Compliance violations reduced by proactive monitoring of regulatory changes.
\end{itemize}



